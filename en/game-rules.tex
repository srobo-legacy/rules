\section {Game Rules}
\label{game-rules}

\begin{enumerate}
\item The game, called \textbf{Pirate Plunder}, will be played in the arena defined in section~\ref{sub:arena}.
      The objective of this game is to achieve as many points as possible by collecting tokens,
       placing them on pedestals, and moving tokens into the team's zone.

\item Before a match starts, the teams participating in that match will be given some time to set their robot up in the arena.
      During this time, they:
\begin{enumerate}
  \item Must place their robot in the zone that they are assigned.
        The robot must be placed such that it is entirely within this zone, with no parts overhanging its boundary.

  \item Must ensure their robot has four robot badges attached it.
        These will be provided by Student Robotics officials at the beginning of the set-up time.
        Section~\ref{sec:robot-badges} provides more information about these markers, as well as their dimensions and mounting requirements.
\end{enumerate}
      Once all robots have been arranged, 20 tokens will be placed in the arena centre.

\item A match lasts 180 seconds.

\item At the end of a match, each team's ``\textbf{game points}'' will be calculated.
      These are used to rank teams before competition league points are awarded.
      Game points will be awarded as follows:
\begin{itemize}
  \item \textbf{1 point} will be awarded for each token that the robot is carrying.
  \item For each token that is entirely within, and in contact with the floor of the team's zone, \textbf{2 points} will be awarded.
  \item \textbf{5 points} will be awarded for each token that is atop the team's pedestal.

\end{itemize}

\item A robot will be considered to be carrying a token if the token's weight is fully supported by the robot, and the token is not in contact with any part of the arena (walls, floor, etc.).
      The judge may ask a member of the robot's team -- or a member of Student Robotics staff -- to lift the robot to demonstrate which tokens are supported by the robot -- if the token moves with the robot, and does not fall off, then it is supported by the robot.

\item At the end of a game, league points will be awarded as follows.
      The team with the \emph{most} game points will be awarded 4 points towards the relevant competition league.
      The team with the second most will be awarded 3.
      The team with the third most will be awarded 2 points, and the team with the fewest game points will be awarded 1 point.
      Teams whose robot was not entered into the round, or who were disqualified from the round, will be awarded no points.

      Tied robots will be awarded the average of the points that their combined positions would be awarded.
      Thus, three robots tied for first place would receive 3 points each (since this is $(4+3+2)/3$).

\item The competition will be structured in two leagues: the \textbf{silver} league, and the \textbf{gold} league.
      Teams will take part in \textbf{6} placement matches each. The upper 5 teams will qualify for gold league, the lower 5 will qualify for silver league.
      Matches will then alternate between leagues, with each team taking part in \textbf{4} matches within that league.
      Finally, the top 4 teams from each league will compete in a final for that league.
      The exact match schedule is documented in Appendix~\ref{apx:matches}.

\item There will be a maximum of 4 robots in a match.
\item Robots will be started by teams leaning into the arena to press the start button on their robot\footnote{A wireless match-starting solution may be provided by Student Robotics} when instructed to do so.

\item There must be no team members in the arena during the 1 minute before a match is scheduled to start.
      Robots must be installed and oriented before this deadline.
      During this minute there must be no interaction with the robot.
      Teams that do not meet this rule will forfeit the match.

\item A match may be terminated prematurely if all teams participating in that match state to the judge that they are happy for the game to end.

\item A token will be considered to be on a pedestal if the token is fully supported by the pedestal, and no part of the token is in contact with a robot, or any other part of the arena.

\end{enumerate}
